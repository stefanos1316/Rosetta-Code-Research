Most empirical studies evaluate software projects
from particular programming language families~\cite{XX}.
A limited number of research works have attempted to compare
features among different programming languages~\cite{XX}.
Here, we count the energy consumption of particular
programming tasks across different programming languages.
To the best of our knowledge,
this is the first study that uses the Rosetta Code data set
to assess the energy consumption in different languages.
In the following, we present related work to our topic,
regarding research on programming languages,
energy consumption, and performance of programming tasks.

% Programming languages
\subsection{Programming Languages}
The study of the strengths and weaknesses of
different programming languages
can help developers to decide {\it which} programming language
they will use to perform particular programming tasks.
For instance, if programmers aim at the scalability
and performance of their systems,
they use functional programming most of the times.
On the other hand, when they want to develop
programs with high modularity,
they use object-oriented programming languages.
In particular, Meyerovich and Rabkin
conducted an empirical study by analyzing
200,000 SourceForge projects
and asking almost 13,000 programmers to
identify characteristics that lead the latter
to select specific programming languages~\cite{MR13}.

In addition, ...

% Energy consumption
\subsection{Energy Consumption}

% Performance
\subsection{Performance}