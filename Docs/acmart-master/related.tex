Most empirical studies evaluate software projects
from particular programming language families.
%A limited number of research works have attempted to compare
%features among different programming languages~\cite{XX}.
Here, we count the energy consumption of
programming tasks across more than ten programming languages.
To the best of our knowledge,
this is the first study that assess
the energy consumption in different programming languages
using the Rosetta Code.
In the following, we present related work to our topic
and compare our results with the results
from previous studies.

% Programming languages
\subsection{Programming Languages}
Studies regarding the strengths and weaknesses of
different programming languages
can help developers to decide {\it which} programming language
they will use to perform specific programming tasks.
For instance, if programmers aim at the scalability
and performance of their systems,
they use functional programming most of the times.
On the other hand, when they want to develop
programs with high modularity,
they use object-oriented programming languages.

Closest to our paper is the empirical study that Nanz and Furia
conducted on the Rosetta Code repository
to compare the efficiency of eight popular programming languages,
including C, Go, C\#, Java, F\#, Haskell, Python, and  Ruby~\cite{NF15}.
Contrary to this work,
we used a power analyzer to run programming tasks
on different programming languages
in order to compare the energy consumption at runtime.

In addition, Meyerovich and Rabkin
conducted an empirical study by analyzing
200,000 SourceForge projects
and asking almost 13,000 programmers to
identify characteristics that lead the latter
to select appropriate programming languages in business level~\cite{MR13}.
However, this study is a survey on the adoption of programming languages in the industry.
Our goal here is different.
We compare the energy consumption of programming tasks
performed in several programming languages.

\subsection{Energy Consumption and Performance}
Several researchers have investigated the energy efficiency
and run-time performance impact over
programming languages.
Also, a significant amount of works have compared the execution
environment where the programs can run efficiently.

In particular, Abdulsalam et al. conducted experiments on workstations~\cite{ALG14},
whereas Rashid et al. used an embedded system~\cite{RAT15} and
Chen and Zong used smart-phones~\cite{CZ16}.
Abdulsalam et al. evaluated the energy effect of four memory
allocation choices ({\tt malloc}, {\tt new}, {\tt array}, and {\tt vector})
and they showed that {\tt malloc} is the most efficient
in terms of energy and performance~\cite{ALG14}.
Chen and Zong showed by using Android Run Time
environment instead of Dalvik,
that the energy and performance implications of Java
are similar to C and C++~\cite{CZ16}.
Rashid et al. compared the energy and performance
impact of four sorting algorithms
written in three different programming languages
({\sc arm} assembly, C/C++, and Java) and
they found that Java consumes the most energy~\cite{RAT15}.
From all these studies it seems that
Java and Python consume a lot of energy and
perform slowly in comparison with C/C++ and assembly.

Additionally, many empirical studies have assessed the impact
of coding practices
(e.g. the use of {\tt for} loops, getters and setters,
static method invocation, views and widgets, and so on)
regarding energy consumption.
Characteristically,
Tonini et al. conducted a study on Android applications
and found that the use of {\tt for} loops with specified length
and the access of class variables without the use of getters and setters
can reduce the amount of the energy that
the applications consume~\cite{TFM13}.
Furthermore, 
in their study, Linares-Vsquez et al. performed analysis
over 55 Android applications from various domains
and they reported
the most energy consuming {\sc api} methods~\cite{LBB14}.
For instance, they found that
the 60\% of the energy-greedy {\sc api}s, 37\% were related to graphical user interface and
image manipulation while the remaining 23\% were associated with the database.

Contrary to previous works,
here we compare energy consuming programming tasks
in more than ten programming languages.
Our results show that ... . % have we found energy consuming tasks in particular programming languages?