\section{Introduction}
%123456789123456789123456789123456789123456789123456789123456789
The increase demands on services and computational applications 
from {\sc ict}-related products are the major facts for contributing 
to increase energy consumption.{\footnote{Although in the physical sense energy 
		cannot be consumed, we will use the terms energy ``consumption'', 
		``requirement'', and ``usage'' to refer to the conversion of 
		electrical energy by {\sc ict} equipment into thermal energy 
		dissipation to the environment. 
		Correspondingly, we will use the term energy ``savings", 
		``reduction", ``efficiency", and ``optimization" to refer 
		to reduced consumption}} 
Recent research by Gelenbe and Caseau~\shortcite{ICT_energy_impact_1} and 
Van Heddeghem et al.~\shortcite{ICT_energy_impact_2} indicates a 
raising trend of the {\sc it}'s sector energy requirements, which are 
expected to reach 15\% of the world's total energy consumption 
by 2020. 
Moreover, obtained results outline, also, raise of green house gas 
emissions due to the {\sc it}-sector which are growing much faster 
than initially predicted and are estimated around 2.3\% globally, as 
claimed in {\sc smart}er2030\footnote{http://smarter2030.gesi.org/downloads.php} 
report.
Therefore, providing energy efficiency at different fields of 
{\sc it} is essential and of paramount importance, since the fact 
is not only economic but also environmental concern. 

%123456789123456789123456789123456789123456789123456789123456789
% Why is importat for software engineering and what are the 
% current trend and goal show the scale of a change...
Traditionally, most of the studies, regarding energy efficiency, 
considered energy consumption at hardware level. 
However, there is much of evidence that software can also 
alter energy dissipation significantly \cite{eder_energy_consumptions, 
capra_is_2012, ferreira_seflab_2013}. 
Therefore, many conferences had identified the energy-efficiency 
at the software level as an emerging research challenge in order 
to reduce energy consumption of a software without comprising its 
run-time performance.


%In fact, some researchers showed by using inefficient data structures 
%or design patterns can increase energy dissipation from 300\% to 700\% 
%respectively \cite{sahin_initial_2012, hasan_energy_profile_for_java_collection_classes_2016}. 
%Nevertheless, even by altering application's energy usage a little 
%bit it can contribute to large scale effect, especially 
%with the large amount of Android smart-phone users 
%amount of available {\sc pc}-like mobile devices, according 
%to Gartner,\footnote{http://www.gartner.com/newsroom/id/3609817} and 
%data-centers.


%123456789123456789123456789123456789123456789123456789123456789
%How to achieve this and 
Nowadays, the development of a software has swifted from traditional 
monolithic architectures and follows a more agile, cutting-edge 
approach such as micro-services. 
The feature of this approach is the development of independent and 
reusable small services in a variety of programming languages.
However, the energy impact of different tasks implemented in different 
programming languages is still new and unknown to researchers and
developers.


In order to identify trends and possible gains for reduce energy 
consumption in software development, we conducted an empirical 
study aiming on eliciting energy usage results starting from small 
tasks implemented in a variety of well-known and highly used 
programming languages. 
To this end, our aim in this research is to identify which 
programming languages offer more energy efficient implementations 
for different tasks.


%123456789123456789123456789123456789123456789123456789123456789
%Paper's remainder.
The remainder of this paper is organized as follows.
In Section~\ref{related_work}, we discuss prior work done in the 
field and compare it with ours. 
Section~\ref{experiment_setup} describes in details our experimental 
platform, the software and the hardware tools we used, how we 
refined our dataset, and our methodology for retrieve our results. 
In Section~\ref{results_and_discussion}, we provide a discussion 
based on our preliminary results and Section~\ref{threats_of_validity} 
details threats to validity.
Finally, we conclude in Section~\ref{conclusiona_and_future_work} 
and we discuss future work and possible directions for this 
research.


\section{Related Work} \label{related_work}

\section{Experiment Setup} \label{experiment_setup}
%123456789123456789123456789123456789123456789123456789123456789
In this Section, we describe the experimental approach to conduct 
our research and retrieve measurements. 
Initially, we provide information about the obtained dataset 
and the way we decided to select our tasks and refine it. 
Moreover, we argue on the selected tasks and programming 
languages we used and the way to refined our dataset.
Furthermore, we explain the setup up of our experimental 
platform, the additional hardware tools, and the software 
tools used to conduct this research. 

\begin{table}
	\begin{threeparttable}
		\caption{Programming Languages, Compilers and Interpreters}
		\label{Languages_Compilers_and_Interpreters}
		\begin{tabular}{cll}
			\toprule
			& Programming  & Compilers and  \\
			& Languages  &  Interpreters version\\
			\midrule
			Compiled&C, C++	& gcc version 6.3.1 20161221- \\
			&		& (Red Hat 6.3.1-1) (GCC) \\
			&Go		& go version go1.7.5  \\
			&Rust	& rustc version 1.18.0 \\
			\midrule
			Semi-&{\sc vb.net} & mono version 4.4.2.0 (vbnc)\tnote{a} \\
			Compiled&C\#		& mono version 4.4.2.0 (mics)\tnote{b} \\
			&Java	& javac version 1.8.0\_131 \\
			\midrule
			Interpreted&JavaScript & node version 6.10.3 \\
			&Perl	& perl version 5.24.1 \\
			&Php		& php version 7.0.19 \\
			&Python	& python version 2.7.13 \\
			&R		& Rscript version 3.3.3 \\
			&Ruby	& ruby version 2.3.3p222 \\
			
			&Swift 	& swift version 3.0.2\tnote{c} \\
			
			\bottomrule
		\end{tabular}
		\begin{tablenotes}
			\begin{small}
				\item[a] {http://www.mono-project.com/docs/about-mono/languages/visualbasic/}
				\item[b] {https://www.codetuts.tech/compile-c-sharp-command-line/}
				\item[c] {https://github.com/FedoraSwift/fedora-swift2/releases/tag/v0.0.2}
			\end{small}
		\end{tablenotes}
	\end{threeparttable}
\end{table}

\begin{table}
		\caption{Optimization Options for Compilers}
		\label{optimization flags}
		\begin{tabular}{lll}
			\toprule
			Programming & Optimization   	& Explanation \\
			Languages  	&  	Option					&\\
			\midrule
			C			& gcc -O3\\
			C++ 		& g++ -O3\\
			Go  		& go run -gcflags  	&Active by default, other-\\
						&	-N				&wise use the -gcflags -N\\
						&					&option to disable it.\\
			Rust		& rustc -O \\
			{\sc vb.net}& vbnc -optimize+,- &+ and - are used to add\\
			C\#			& mcs -optimize+,- 	& or remove optimization\\
						&					& respectively.\\
			Java		& java Xint			&Used to execute byte-code\\	
						&					&in interpreted mode\\	
			\bottomrule
		\end{tabular}
\end{table}

\subsection{Dataset}
%123456789123456789123456789123456789123456789123456789123456789
In the context of this study, we used Rosetta 
Code,\footnote{http://rosettacode.org/wiki/Rosetta\_Code} a 
publicly available programming chrestomathy site that offers 
851 tasks, 230 draft tasks, and a collection of 658 different 
programming languages. In general, not all of (and cannot) tasks 
are implemented in all languages. 
We found and downloaded a Github repository\footnote{https://github.com/acmeism/RosettaCodeData} 
which contains all the currently implemented tasks introduced in 
Rosetta Code website.

%123456789123456789123456789123456789123456789123456789123456789
For selecting the highly used programming languages, we made use 
of tiobe,\footnote{https://www.tiobe.com/tiobe-index/} a software 
quality company.
By making use of a formula,\footnote{https://www.tiobe.com/tiobe-index/programming-languages-definition/} 
25 of the highest ranked search engines (according to Alexa),\footnote{http://www.alexa.com/} 
and a number of requirements enlisted for programming languages, 
tiobe provides a search query for index rating of the most popular 
programming languages around the web for each month. 
Initially, we decided of choosing the top 15 programming languages 
as enlisted for June 2017. 
From the current list, we excluded programming languages such as 
Delphi and Assembly. 
In contrast, we included Rust in our dataset which is a memory safe 
programming language and is gaining vast popularity in the web. \
Therefore, we ended up with 14 programming languages as illustrated 
in Table~\ref{Languages_Compilers_and_Interpreters}.

%123456789123456789123456789123456789123456789123456789123456789 
In terms of selecting tasks, we developed a shell script (more 
details in Subsection~\ref{software_components}) to identify 
which of the 851 tasks offer the most implementations for the 
programming languages of our selection.
After launching our script, we obtained around 29 different tasks. 
For the context of our preliminary study, we choose only 
nine tasks implemented in the most of the programming languages 
of our selection. 
The selected dataset tasks were \textit{array-concatenation, classes} 
(creating an object and calling a method to print a variable's value), 
\textit{url-encoding and decoding, bubble, quick, insertion, merge, 
	and selection sorting algorithms}
Moreover, to refine further our dataset we used the following steps: 

\begin{enumerate}
	\item [$\bullet$] Some of the tasks offered more that one implementation 
	for the same programming languages. 
	Thus, we had to drive manually through each directory and remove 
	them until we have only one that is consistent with the other 
	implementation. 
	For example, when most of the implemented tasks used iterative implementation 
	we removed the ones using recursion. 
	\item [$\bullet$] The Java file's names were 
	different from the public class names which results to compilation 
	error if not changed accordingly. 
	\item [$\bullet$] Some of the implementations didn't have main 
	classes, nor the same data with other tasks. 
	Therefore, we change the source code to offer consistency.
	\item [$\bullet$] For some programming languages which do not offer 
	the class option such as C and Go, we used structs. 
	\item [$\bullet$] Some of the tasks are relatively small and may 
	finish faster than a second which makes it impossible for our 
	power analyzer to capture those results. 
	Therefore, we added all the selected tasks in an iteration loop of 
	a million times. 
\end{enumerate}

After applying the above modification on our dataset, we categorized our 
programming languages in three main categories as illustrated in 
Table~\ref{Languages_Compilers_and_Interpreters}. 
Moreover, for the programming languages which offer a compiled approach 
such as Java, {\sc vb.net}, and C\#, we added them under the category of 
compiled languages. 
In addition, we compared the compiled and semi-compiled implementations while 
using the available compiler optimization and without them, as showed in 
Table~\ref{optimization flags}.



\begin{figure*}
	\centering
	\includegraphics[width=\textwidth,height=8cm]{"Graph_Experiment-Data-Sun-Jul-9-19:14:04-UTC-2017_compiled_Optimization_OFF"}
	\caption{Results for Compiled Programming Languages without Optimization Option}
	\label{Compiled with No Optimization}
\end{figure*}

\begin{figure*}
	\centering
	\includegraphics[width=\textwidth,height=8cm]{"Graph_Experiment-Data-Mon-Jul-10-07:03:37-UTC-2017_compiled_Optimization_On"}
	\caption{Results for Compiled Programming Languages with Optimization Option}
	\label{Compiled with Optimization Option}
\end{figure*}

\subsection{Hardware and Software components}

\begin{table}
	\begin{threeparttable}
		\caption{System hardware and software specifications}
		\label{laptop_specs}
		\begin{tabular}{cl}
			\toprule
			&Description\\
			\midrule
			Hardware	& \textbf{HP EliteBook 840 G3}, Intel Core i7-6500U \\
			& (2 physical cores of 2.5 GHz), 8 GB DDR4  \\
			& memory, 256 GB {\sc ssd}  hard disk, \\
			& \textbf{Raspberry Pi Model 3b}, 4x ARM A53 \\
			& 1.2GHz, 2 GB LPDDR2 memory, 64 GB SD\\
			Operating  System & Fedora 25 kernel version 4.11.5-200,  \\
			& Raspbian\\	
			Software 	& {\sc wup} software (retrieving measurements \\
			& from the device), bash script, Gnuplot 5.0 \\
			\bottomrule
		\end{tabular}
	\end{threeparttable}
\end{table}

\subsubsection{Hardware Components}
%123456789123456789123456789123456789123456789123456789123456789 
The physical tools composed mainly from a portable personal 
computer, a real-time electricity usage monitoring tool, and 
an embedded device.
Our systems' specifications are depicted in Table~\ref{laptop_specs}. 
The real-time power usage tools we used is the Watts Up Pro ({\sc wup}).\footnote{https://www.wattsupmeters.com/secure/products.php?pn=0} 

In general, there are two venues for retrieving energy consumption 
from a computer-based system: on one hand, by indirect energy measurements 
through estimation models or performance counters, core component 
of software monitoring tools, and on the other hand, via direct measurement, 
hardware power analyzers and sensors.  
However, each approach has it own pitfalls such as: coarse-grained 
measurements for the whole systems' energy consumption and low sampling 
rate for direct measurements case and inaccuracy, lack of interoperability, 
and additional system overhead while using indirect measurements. 
Therefore, in our research we decided to retrieve our energy consumption 
measurements using direct approach such as {\sc wup} since our 
tasks are relatively short in terms of source-code lines.

In regards to {\sc wup}, it offers accuracy of \textpm1.5\% and 
as minimum sampling rate of a second. 
In order to retrieve power-related measurements from the {\sc wup} 
we used a Linux-based interface utility available in a Git 
repository.\footnote{https://github.com/pyrovski/watts-up}
This software helped us retrieve measurements such as timestamps, 
watts, volts, amps, etc. through a mini {\sc usb} interface after 
we integrated its code in our script that runs all the tasks. 
In order to avoid additional overhead in our measurements, we 
used a Raspberry Pi\footnote{https://www.raspberrypi.org/products/raspberry-pi-3-model-b/} 
to retrieve power consumption from our test-bed.  


\begin{figure*}
	\centering
	\includegraphics[width=\textwidth,height=8cm]{"Graph_Experiment-Data-Mon-Jul-10-07:03:37-UTC-2017_interpreted-1"}
	\caption{Results for Interpreted Programming Languages}
	\label{Interpreted}
\end{figure*}

\subsubsection{Software Components} \label{software_components}
%123456789123456789123456789123456789123456789123456789123456789 
To extract data, manage, and use our Rosetta Code Repository, 
we developed a number of shell scripts as enlisted below and are 
publicly available on our Git repository.\footnote{https://github.com/stefanos1316/Rosetta-Code-Research} 

\begin{enumerate}
	\item [$\bullet$] \textbf{script.cleanAll}, removes the current instance 
	of Tasks in the current working directory and copy the new one found 
	from in the parent directory. 
	\item [$\bullet$] \textbf{script.findCommonTasksInLanguages}, provides 
	a list of tasks with the amount of programming language implementations.
	\item [$\bullet$] \textbf{script.createNewDataSet}, filters the Rosetta 
	Code current dataset and removes programming languages and tasks not 
	added as command line arguments.
	\item [$\bullet$] \textbf{script.fromUpperToLower}, changes the current 
	instance of Tasks directory tree's letters from upper to lower 
	case to offer consistency for our scripts. 
	\item [$\bullet$] \textbf{script.compileTasks}, compiles all tasks found 
	under the Tasks' directory and produces error reports in a task fails to 
	compile.
	\item [$\bullet$] \textbf{script.executeTasksRemotely}, executes all the tasks' 
	implementations found in under Tasks directory. 
	Moreover, it sends command to {\sc wup}, retrieves measurements and stores 
	then on remote host, through \textit{ssh}, in order to start retrieving 
	measurements for each test case. 
	\item [$\bullet$] \textbf{script.createPlottableData}, creates a single file 
	that enlists all the executed tasks with the energy consumption for each 
	implementation. In addition, we used {\sc ntp}\footnote{http://www.ntp.org/} to 
	synchronize both systems' clock which helped us to map our results of 
	run-time performance and energy consumption. 
	\item [$\bullet$] \textbf{script.plotGraphs}, after retrieving our data 
	we use this script to plot our graphs. 
	For plotting our graphs we used Gnuplot,\footnote{http://www.gnuplot.info/} 
	an open-source general purpose pipe-oriented plotting tool.
\end{enumerate}

%123456789123456789123456789123456789123456789123456789123456789 
Note that most of the scripts offer the \textit{--help} option 
that shows a list of available command line arguments and options.
In addition, we provide a {\sc readme.md} file, available in our 
repository, as a guideline for using our scripts and reproducing 
the obtained results. 
We tried to automate the execution procedure as much as possible 
in order to remove the burden from users who would like to use 
our scripts.
Moreover, we suggest for the users not to change the directory's names or 
locations since it will alter the correct sequence of the 
execution. 


\subsection{Retrieving Energy Measurements} 
%123456789123456789123456789123456789123456789123456789123456789 
As an initial step for our experiment, we shut down background 
process, as suggested by Hindle~\shortcite{hindle_greenminer:_2014}, 
found in modern {\sc os} (Operating System) such as disk defragmentation, virus 
scanning software, {\sc cron} jobs, automatic updates, 
disk indexing, document indexing, {\sc rss} feed updates, etc. to 
minimize possible noise interferences in our measurements. 
By making the following steps, we reduced our platform's idle 
power consumption from 8,6 to 5.8 watts in average. 


%123456789123456789123456789123456789123456789123456789123456789 
We estimated after an {\sc os} is launched, 
it's necessary to wait for a short period to reach a 
\textit{stable condition} which is close to five minutes \cite{carroll_analysis_2010}.
After reaching \textit{stable condition}, we launched our main 
script \textit{i.e.,} \textbf{script.executeTasksRemotely}, 
that executes all the tasks implemented in different programming 
languages. 
Before executing a tasks, the execution script sends a command 
to the remote host ,\textit{i.e.,} Raspberry Pi, through a 
password-less {\sc ssh} connection to starting 
collecting power consumption measurements from {\sc wup} 
for the currently executing task. 
In addition, the local host retrieves run-time performance 
measurements through command 
time\footnote{https://linux.die.net/man/1/time} and stores them 
in timestamped directories which we analyze later.
Between each execution of a task, we added a 
sleep\footnote{http://man7.org/linux/man-pages/man3/sleep.3.html} 
period of three minutes. 
The time gap exist to ensure that our experimental platform 
reached a s\textit{table condition}, to avoid unnecessary noise 
in our measurements. 
For example, to ensure the platform's {\sc cpu} is cooled down 
and the fan is no longer consuming more power.

\section{Results and Discussion} \label{results_and_discussion}
%123456789123456789123456789123456789123456789123456789123456789 
For this Section, we analyze and discuss the collected data after 
running our experiments. 
The results in Figure~\ref{Compiled with No Optimization} shows 
a relatively small diverge in energy usage without the use of 
compiler optimization. 
The obtained results shows that Java introduced the highest 
energy consumption among the compiled programming languages. 
In the case of Java, we used the \textit{-Xint} flag that runs 
the byte-code as interpreted and avoids the use of 
Just-in-Time\footnote{https://docs.oracle.com/cd/E15289\_01/doc.40/e15058/underst\_jit.htm} 
compiler.

%123456789123456789123456789123456789123456789123456789123456789 
Figure~\ref{Compiled with Optimization Option} depicts the average 
energy-related results with the option of compiler's optimization. 
The results illustrate energy reduction, in most cases, after 
applying the optimization option it exhibits energy savings. 
In many cases, the results in Figure~\ref{Compiled with Optimization Option} 
tend to zero energy consumption.
A fact that stems from the short tasks' execution time, \textit{i.e.,} 
less than a second, which makes in impossible for the {\sc wup} analyzer 
to retrieve energy measurements at this time interval.
Also, C and C++ are the programming languages that offer the most 
of energy gains after the \textit{-O3} optimization flag 
is applied. 
Moreover, Java, {\sc vb.net}, and C\# offers the most inefficient 
energy measurements, in average, for almost all tasks with the 
optimization option. 
A possible reason of the outcome is the use of virtual machines to 
run the byte-code, memory management, and the garbage collector.

%123456789123456789123456789123456789123456789123456789123456789 
For the interpreted programming languages, we can see the energy 
consumption scales similarly for all the tasks. 
On one hand, we see that R and Python results to the highest energy 
dissipation for all tasks except from \textit{Classes}. 
On the other hand, JavaScript and Swift contributes to the lowest 
energy usage except for the \textit{Classes} task. 
Moreover, Ruby's execution time is less than a second for the 
\textit{array-concatenation} task, thus, it's energy usage is 
depicted as zero.


\section{Threats to validity} \label{threats_of_validity}
%123456789123456789123456789123456789123456789123456789123456789 
\noindent\textbf{Internal:} In order to avoid additional overhead 
on our experimental platform, we used a remote host to collect our 
results. 
Therefore, the need of wireless connection was necessary, which 
might incur in additional energy requirements by making use of 
the {\sc ssh} to start and stop the {\sc wup}. 
Moreover, we cannot have full control of our {\sc os} workloads 
and background operations, therefore, is possible that some 
daemons might start running while test our experiment.
  
\noindent\textbf{External:} Our real-time power analyzer offers minimum sampling 
interval of a second. 
Therefore, in Figures~\ref{Compiled with No Optimization}, 
~\ref{Compiled with Optimization Option}, and ~\ref{Interpreted} 
the energy dissipation resulting to zero are interpreted 
as the tasks execution to be less than a seconds, which makes in 
impossible for {\sc wup} to capture such measurements.

\section{Conclusion and Future Work} \label{conclusiona_and_future_work}
%123456789123456789123456789123456789123456789123456789123456789 
To this end we conclude as follows. 
The average energy consumption of compiled programming languages is 
much lower compared to the interpreted ones for the tested tasks. 
Except for a single case, were JavaScript consumes less energy for 
the \textit{Selection-Sort} algorithm. 
In addition, optimization option shows significant results are 
available for almost all the test cases.

More specifically, Swift and Ruby offers the most energy gains 
for \textit{sorting} tasks and \textit{url encoding} and 
\textit{decoding}.
In respect to compiled programming languages with the optimization 
option C and  C++ contributes the lowest energy usage for tasks 
such as sorting algorithms.

%123456789123456789123456789123456789123456789123456789123456789 
As for future work, we would like to test all the 29 collected 
tasks and, furthermore, to developed more such as exception 
and task for functional programming. 
Moreover, we will test the collected tasks in different {\sc cpu} 
architectures such as {\sc amd} and {\sc arm}. 
In addition, we plan to collect resource usage to identify 
possible relationship between programming  languages and 
resources. 
To this end, we expected the obtained results to shed light 
and provide further understanding on developing in an energy 
efficient manner for larger and  more complex applications.



